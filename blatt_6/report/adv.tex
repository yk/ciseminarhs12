\documentclass[11pt,a4paper]{article}
\usepackage[latin5]{inputenc}
\usepackage[english]{babel}
\usepackage{amsmath}
\usepackage{amsfonts}
\usepackage{amssymb}
\usepackage{graphicx,subfig}
\usepackage{placeins}
\usepackage{gensymb}

\author{Alexander Attinger, Yannic Kilcher}
\title{Report Blatt 6, Advanced Part}

\begin{document}
\maketitle

\section{Affine Fundamental Matrix}
\paragraph{a}
It is to be shown that all epipolar lines resulting from an affine fundamental matrix are parallel.

Our affine fundamental matrix will be:

\[
\begin{array}{lc}
F_{A} = & \left(\begin{array}{@{}ccc@{}}
                    0 & 0 & a \\
                    0 & 0 & b \\
                    e & d & c
                  \end{array}\right)\\ [15pt]
  
\end{array} 
\]

If we calculate an epipolar line $l=Fx$ with $x=(u,v,1)$, we obtain $l=(a,b,c+dv+eu)$.\\
We can form this into the line equation $ai+bj+(c+dv+eu) = 0$ ($i$ and $j$ being the new image coordinates) and finally to $j=(a/b)i + (c+dv+eu)/b$ with the elevation of the line ($a/b$) being independent of $u$ and $v$.
Therefore, for all input $x$, this line will have the same elevation and therefore, all these lines are parallel.

\paragraph{b}
We have used the Zisserman algorithm to estimate the affine fundamental matrix with the correspondances from the basic part of the exercises.
For that, we centered the aligned correspondances and then calculated the SVD on them. The parameters of the fundamental matrix can then be obtained
from the singular vector corresponding to the smallest singular value.

Our affine fundamental matrix turned out to be:

\[
\begin{array}{lc}
F_{A} = & \left(\begin{array}{@{}ccc@{}}
                    0 & 0 & -0.792 \\
                    0 & 0 & -0.110 \\
                    0.579 & 0.162 & 23.244
                  \end{array}\right)\\ [15pt]
  
\end{array} 
\]

\paragraph{c}
We have used the RANSAC algorithm to produce many estimates of the fundamental matrix by using only 5 random correspondances for each estimate.
We have kept the one having the best score, which turned out to be (after 1000 trials).

\[
\begin{array}{lc}
F_{A} = & \left(\begin{array}{@{}ccc@{}}
                    0 & 0 & -0.275 \\
                    0 & 0 & -0.847 \\
                    0.425 & 0.160 & -22.763
                  \end{array}\right)\\ [15pt]
  
\end{array} 
\]


\end{document}
